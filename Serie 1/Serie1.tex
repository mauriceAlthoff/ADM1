\documentclass[a4paper,10pt,german]{scrartcl}
\usepackage[T1]{fontenc}
\usepackage[utf8]{inputenc}
\usepackage[ngerman]{babel}
\usepackage{dsfont,alltt}
\usepackage{amsthm,amssymb,amsmath}
\usepackage{tikz}
\usetikzlibrary{arrows}
\newcommand{\komp}[1]{\ensuremath{\overline{#1}}}
\newcommand{\gdw}{\ensuremath{\Leftrightarrow}}
\newcommand{\dann}{\ensuremath{\Rightarrow}}
\newcommand{\K}{\ensuremath{\mathbb{K}}}
\newcommand{\N}{\ensuremath{\mathbb{N}}}
\newcommand{\one}{\ensuremath{{\mathds{1}}}}


\newcommand{\divec}[2]{\ensuremath{\left(\begin{array}{cc}
                                    #1\\
                                    #2
                                   \end{array}\right)
                                   }}

\DeclareMathOperator*{\cone}{cone}
\DeclareMathOperator*{\conv}{conv}
\DeclareMathOperator*{\aff}{aff}

\newenvironment{ggbsp}{\begin{proof}[Gegenbeispiel]}{ \end{proof}}

\renewcommand{\labelenumi}{\alph{enumi}\,)}

%opening
\title{Einführung in die lineare und kombinatorische Optimierung\\
Serie 1}
\author{Maurice Althoff (FU 4745454)\\Michael R. Jung (HU 502133)\\Felix Völker (TU 331834)}

\begin{document}

\maketitle

\section*{Aufgabe 1}
\begin{enumerate}
\item\ \\\begin{center}
  \begin{tabular}{| l | c | c | c | c |}
    \hline
                & Montagehalle(4) & Fertigung(4) & Qualit\"atskontrolle(2) & Preis\\ \hline
    Turamichele &  6 min          &  2 min       &   4 min                 & 50 Euro\\ \hline
    Jim Knopf   & 10 min          &  4 min       &   4 min                 & 60 Euro \\ \hline
    Arbeitszeit & 45 min/h        & 50 min/h     &  50min/h                & \\ \hline    
  \end{tabular}
\end{center}

\begin{tikzpicture}[->,>=stealth',shorten >=1pt,auto,node distance=3cm,
  thick,main node/.style={circle,draw,font=\sffamily\Large\bfseries}]

  \node[main node] (M) {M};
  \node[main node] (F) [right of=M] {F};
  \node[main node] (Q) [right of=F] {Q};
  \node[main node] (A) [right of=Q] {A};

  \path[every node/.style={font=\sffamily\small}]
    (M) edge [bend left]  node {6 min} (F)
        edge [bend right] node {10 min} (F)
    (F) edge [bend left]  node {2 min} (Q)
        edge [bend right] node {4 min} (Q)
    (Q) edge [bend left]  node {4 min} (A)
        edge [bend right] node {4 min} (A);
\end{tikzpicture}

Kosten: $max 6f_{MPT} + 2 f_{FQT} + 4f_{QAT} + 10f_{MPJ} + 4 f_{FQJ} + 4f_{QAJ}$\\
Variablen: \\
$0 \leq f_{MPT} \leq 4$; $0 \leq f_{MPJ} \leq 4$\\
$0 \leq f_{FQT} \leq 4$; $0 \leq f_{FQJ} \leq 4$\\
$0 \leq f_{QAT} \leq 2$; $0 \leq f_{QAJ} \leq 2$\\
$6f_{MFT} + 10f_{MFJ} \leq 45 min$\\
$2f_{FQT} + 4f_{FQJ} \leq 50 min$\\
$4f_{QAT} + 4f_{QAJ} \leq 50 min$
\end{enumerate}
\section*{Aufgabe 2}
Formulierung unter Zuhilfenahme der Graphentheorie:\\
Z.z.: Für alle Graphen $G=(V,E)$ Graph mit $|V|\geq6$ gilt: $\exists V'\subset V:G[V']\cong K_3\vee G[V]\cong \komp{K_3}$.
\begin{proof}
 Sei $G=(V,E)$ ein Graph mit $|V|\geq6$.\\
 Betrachte einen Knoten $v$ und 5 weitere Knoten $x_1,\dots,x_5$. Unter diesen ist er entweder mit mindestens dreien benachbart oder mit mindestens dreien nicht benachbart.
 \begin{itemize}\itemindent25pt
  \item[{\bf Fall 1:}]$v$ ist mit drei Knoten benachbart (o.B.d.A. seien dies $x_1,x_2,x_3$):\\
  Entweder existiert unter diesen mindestens eine Kante (o.B.d.A. $\{x_1,x_2\}$), dann ist $G[\{v,x_1,x_2\}]\cong K_3$ oder unter diesen existiert überhaupt keine Kante, 
  dann gilt aber $G[\{x_1,x_2,x_3\}]\cong\komp{K_3}.\hfill \checkmark$
  \item[{\bf Fall 2:}]$v$ ist mit drei Knoten nicht benachbart (o.B.d.A. seien dies $x_1,x_2,x_3$):\\
  Entweder existiert unter diesen mindestens eine Kante nicht (o.B.d.A. $\{x_1,x_2\}$), dann ist $G[\{v,x_1,x_2\}]\cong \komp{K_3}$ oder unter diesen existieren alle Kanten, 
  dann gilt aber $G[\{x_1,x_2,x_3\}]\cong K_3.\hfill \checkmark$
 \end{itemize}
\end{proof}

\section*{Aufgabe 3}
Sei $G=(V,E)$ ein Graph. Es genügt zu zeigen: $$G\text{ ist nicht zusammenhängend }\Rightarrow \komp{G}\text{ ist zusammenhängend.}$$
\begin{proof}
Sei $G=(V,E)$ ein unzusammenhängender Graph mit den Zusammenhangskomponenten $K_1,\dots,K_p\subseteq V,\ \bigcup\limits_{i=1}^pK_i=V$. Seien $x,y\in V$ beliebig.
\begin{itemize}\itemindent25pt
 \item[{\bf Fall 1:}] Es existieren $i\neq j\in\{1,\dots,p\}$ mit $x\in K_i\wedge y\in K_j$:\\
 Nun gilt: $\{x,y\}\notin E\Rightarrow \{x,y\}\in E(\komp{G})=:\komp{E}.\hfill\checkmark$
 \item[{\bf Fall 2:}]Es existieren $z\in V, i\neq j\in\{1,\dots,p\}$ mit $x,y\in K_i\wedge z\in K_j$ (da $G$ nicht zusammenhängend ist muss es Knoten in anderen Komponenten geben):\\
 $\Rightarrow\{x,z\},\{y,z\}\notin E\Rightarrow \{x,z\},\{y,z\}\in\komp{E}\Rightarrow\text{Der Weg } (x,z,y)\text{ existiert in }\komp{G}.\hfill\checkmark$
\end{itemize}
Da dies für je zwei Knoten in $V$ gilt, ist $\komp{G}$ zusammenhängend. 
\end{proof}
\pagebreak
\section*{Aufgabe 4}
\begin{enumerate}
 \item Die Aussage $\cone(S\cup T)=\cone(S)+\cone(T)$ ist korrekt.
  \begin{proof}
   \begin{align*}
    &x\in\cone(S\cup T)\\
    \gdw&\exists \lambda_1,\dots\lambda_k,\alpha_1,\dots,\alpha_n\geq0\in\mathbb{K},s_1,\dots,s_k\in S,t_1,\dots,t_n\in T:\\ 
    &x=\sum\limits_{i=1}^k\lambda_is_i+\sum\limits_{i=1}^n\alpha_it_i\\
    \gdw&\exists \lambda_1,\dots\lambda_k,\alpha_1,\dots,\alpha_n\geq0\in\mathbb{K},s_1,\dots,s_k,s'\in S,t_1,\dots,t_n,t'\in T:\\ 
    &x=s'+t'\wedge s'=\sum\limits_{i=1}^k\lambda_is_i\wedge t'=\sum\limits_{i=1}^n\alpha_it_i\\
    \gdw&\exists  s'\in\cone(S),t'\in\cone(T): x=s'+t'\\
    \gdw&x\in\cone(S)+\cone(T)
   \end{align*}
  \end{proof}
 \item Die Aussage $\aff(S\cup T)= \aff(S)+\aff(T)$ ist inkorrekt.
  \begin{ggbsp}
   Seien $S=\left\{\divec{0}{1}\right\},T=\left\{\divec{1}{0}\right\}$.\\
   Dann ist $\aff(S)=S,\aff(T)=T,\aff(S)+\aff(T)=\left\{\divec{1}{1}\right\}$,\\
   aber $\aff(S\cup T)=\left.\left\{\divec{\alpha}{1-\alpha}\right|\alpha\in\K\right\}$.
  \end{ggbsp}
 \item Die Aussage $\aff(S+T)= \aff(S)+\aff(T)$ ist korrekt.
  \begin{proof}
   \begin{itemize}
    \item[{,,$\subseteq$}'']
     \begin{align*}
      &\text{Sei }x\in\aff(S+T)\\ 
      \dann&\exists k\in\N,\lambda\in \K^k\text{ mit }\lambda^T\one=1,x_1,\dots,x_k\in S+T:\\
      &x=\sum\limits_{i=1}^k\lambda_i x_i\\
      \stackrel{\text{Def. }S+T}{\dann}&\exists k\in\N,\lambda\in \K^k\text{ mit }\lambda^T\one=1, s_1,\dots,s_k\in S,t_1,\dots,t_k\in T:\\
      &x=\sum\limits_{i=1}^k\lambda_i(s_i+t_i)\\
      \dann&\exists k\in\N,\lambda\in \K^k\text{ mit }\lambda^T\one=1, s_1,\dots,s_k\in S,t_1,\dots,t_k\in T:\\
      &x=\sum\limits_{i=1}^k\lambda_is_i+\sum\limits_{i=1}^k\lambda_it_i\\
      \stackrel{\text{Def. }\aff}\dann&\exists s'\in\aff(S),t'\in \aff(T):x=s'+t'\\
      \dann&x\in\aff(S)+\aff(T)
     \end{align*}
    \item[{,,$\supseteq$}'']
     \begin{align*}
      &\text{Sei }x\in\aff(S)+\aff(T)\\
      \dann\exists& k,l\in\N,\alpha\in \K^k,\beta\in\K^l\text{ mit }\alpha^T\one=1,\beta^T\one=1,s_1,\dots,s_k\in S,t_1,\dots,t_l\in T:\\
      x&=\sum\limits_{i=1}^k\alpha_i s_i+\sum\limits_{i=1}^l\beta_i t_i\\
      \intertext{ \footnotesize Sei o.B.d.A.  $k=l$ ($k<l\dann\alpha_{k+1}=\dots=\alpha_l=0, s_{k+1}=\dots=s_l=s_k,\,k>l$ analog).}
      \dann\exists& k\in\N,\alpha,\beta\in \K^k\text{ mit }\alpha^T\one=1,\beta^T\one=1,s_1,\dots,s_k\in S,t_1,\dots,t_k\in T:\\
      x&=\sum\limits_{i=1}^k\alpha_i s_i+\sum\limits_{i=1}^l\beta_i t_i\\
      &=\sum\limits_{i=1}^k\left(-\frac1k(s_i+t_i)+\underbrace{\sum\limits_{j=1}^k\frac{\alpha_i+\beta_j}k(s_i+t_j)}
      _{\text{insges.:\,}\sum\limits_{m=1}^k(\alpha_m+\frac1k)s_m+(\beta_m+\frac1k)t_m}\right)\\
      \intertext{Setze $u_{i,j}:=s_i+t_j \in S+T,\gamma_{i,j}:=\frac{\alpha_i+\beta_j}{k},\delta_i=-\frac1k'\in\K$, wobei gilt:}
      &\ \sum\limits_{i=1}^k\sum\limits_{j=1}^k\gamma_{i,j}+\sum\limits_{i=1}^k\delta_i=\sum\limits_{i=1}^k\left(\alpha_i+\frac1k\right)-1=1+1-1=1\\
      \intertext{Es gilt also:}
      x&=\sum\limits_{i=1}^k\sum\limits_{j=1}^k\gamma_{i,j}u_{i,j}+\sum\limits_{i=1}^k\delta_iu_{i,i}
      \intertext{und dies ist eine affine Kombination aus $S+T$.}
     \end{align*}
   \end{itemize}
  \end{proof}
 \item Die Aussage $\conv(S\cup T)= \conv(S)+\conv(T)$ ist inkorrekt.
  \begin{ggbsp}
   Seien $S=\left\{\divec{0}{1}\right\},T=\left\{\divec{1}{0}\right\}$.\\
   Dann ist $\conv(S)=S,\conv(T)=T,\conv(S)+\conv(T)=\left\{\divec{1}{1}\right\}$,\\
   aber $\conv(S\cup T)=\left.\left\{\divec{\alpha}{1-\alpha}\right|\alpha\in\K\cap[0,1]\right\}$.
  \end{ggbsp}
 \item Die Aussage $\conv(S+T)= \conv(S)+\conv(T)$ ist korrekt.
 \begin{proof}
 
 Betrachte Algorithmus $coeff((\alpha),(\beta),(s),(t)$ mit folgenden Eingaben:
 \begin{itemize}
  \item Zwei Folgen $(\alpha_i)_{1\leq i\leq k},\alpha_i>0$ und $(\beta_i)_{1\leq i\leq l},\beta_i>0$ mit $\sum\limits_{i=1}^k\alpha_i=\sum\limits_{j=1}^m\beta_j=1$ für ein $c\in\K$.
  \item Zwei Folgen $(s_i)_{1\leq i\leq k},s_i\in S,(t_i)_{1\leq i\leq l},t_i\in T$.
 \end{itemize}
 \begin{alltt}coeff:
global i=1
Falls k,l\(\geq 1)\)
  begin
    Falls \(\alpha_1>\beta_1\)
      begin
        \(\gamma_i:=\alpha_1\)
        \(u_i:=s_1+t_1\)
        \(\beta_1:=\beta_1-\alpha_1\)
        coeff(\((\alpha_2,\dots,\alpha_k),(\beta),(s_2,\dots,s_k),(t))\)
      end
    sonst Falls \(\alpha_1<\beta_1\)
            begin
              \(\gamma_i:=\beta_1\)
              \(u_i=s_1+t_1\)
              \(\alpha_1:=\alpha_1-\beta_1\)
              coeff(\((\alpha),(\beta_2,\dots,\beta_l),(s),(t_2,\dots,t_l))\)
            end
         sonst
            begin
              \(\gamma_i:=\beta_1        (=\alpha_1)\)
              \(u_i=s_1+t_1\)
              coeff(\((\alpha_2,\dots,\alpha_k),(\beta_2,\dots,\beta_l),(s_2,\dots,s_k),(t_2,\dots,t_l))\)
            end
  i:=i+1
  end
\end{alltt}
Da beide Folgen zu $1$ aufsummieren, kann es nicht passieren, dass man mit der einen Folge früher fertig ist. Daher ist das Ergebnis wohldefiniert.
Es gilt nun: $$\sum\limits_m(\gamma_mu_m)=\sum\limits_i(\alpha_is_i)+\sum\limits_j(\beta_jt_j)$$
und $\sum\limits_m(\gamma_mu_m)$ ist eine konvexe Kombination in $S+T$.
 \end{proof}
\end{enumerate}

\end{document}
