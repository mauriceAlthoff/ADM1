\documentclass[a4paper,10pt,german]{scrartcl}
\usepackage[T1]{fontenc}
\usepackage[utf8]{inputenc}
\usepackage[ngerman]{babel}
\usepackage{a4wide}
\usepackage{dsfont,alltt}
\usepackage{amsthm,amssymb,amsmath,ifthen}
\usepackage{tikz}
\usetikzlibrary{arrows}
\newcommand{\komp}[1]{\ensuremath{\overline{#1}}}
\newcommand{\gdw}{\ensuremath{\Leftrightarrow}}
\newcommand{\dann}{\ensuremath{\Rightarrow}}
\newcommand{\K}{\ensuremath{\mathbb{K}}}
\newcommand{\N}{\ensuremath{\mathbb{N}}}
\newcommand{\R}{\ensuremath{\mathbb{R}}}
\newcommand{\one}{\ensuremath{{\mathds{1}}}}



\newcounter{ctra}
\newcommand{\hilfsstring}{}                             
% \vect[ausrichtung]{n}{text} erzeugt eine Matrix mit runden Klammern der Breite n
% text muss dann & und \\ enthalten, z.B. 11&12&13\\21&22&23
% ausrichtung ist optional und richtet alle Spalten aus, default ist c
\newcommand{\vect}[3][c]{\setcounter{ctra}{0}\renewcommand{\hilfsstring}{} \whiledo{\value{ctra}<#2}{\edef\hilfsstring{#1\hilfsstring}\stepcounter{ctra}}
\ensuremath{\left(\begin{array}{\hilfsstring}#3\end{array}\right)}}
\DeclareMathOperator*{\cone}{cone}
\DeclareMathOperator*{\conv}{conv}
\DeclareMathOperator*{\aff}{aff}

\newenvironment{ggbsp}{\begin{proof}[Gegenbeispiel]}{ \end{proof}}

\renewcommand{\labelenumi}{\alph{enumi}\,)}

%opening
\title{Einführung in die lineare und kombinatorische Optimierung\\
Serie 2}
\author{Maurice Althoff (FU 4745454)\\Michael R. Jung (HU 502133)\\Felix Völker (TU 331834)}

\begin{document}

\maketitle

\section*{Aufgabe 5}
 \begin{enumerate}
  \item
   \begin{align*}
     x_1\dots\,&\text{Anzahl produzierter Müsli-Packungen vom Typ A}\\
     x_2\dots\,&\text{Anzahl produzierter Müsli-Packungen vom Typ B}\\
     x_3\dots\,&\text{Anzahl produzierter Müsli-Packungen vom Typ C}\\
     c:=&\vect1{5\\4\\3} b:=\vect[r]1{5000\\11000\\8000}x:=\vect1{x_1\\x_2\\x_3} A:=\vect3{2&3&1\\4&1&2\\3&4&2}\\
     \text{LP }P:\ \max\ &c^Tx\hspace*{20pt} \text{unter den Nebenbedingungen}\\
     Ax&\leq b\\
     x&\geq 0
   \end{align*}
    ausgeschrieben:
    \begin{align*}
   \max\ &5x_1+4x_2+3x_3 &&\text{unter den Nebenbedingungen}\\
    &2x_1+3x_2+1x_3\leq 5000&&&(y_1)\\
    &4x_1+1x_2+2x_3\leq 11000&&&(y_2)\\
    &3x_1+4x_2+2x_3\leq 8000&&&(y_3)\\
    &x_1,x_2,x_3\geq 0
    \end{align*}
   \item
    \begin{align*}
     D:\ \min\ &b^Ty \hspace*{20pt} \text{unter den Nebenbedingungen}\\
     A^Ty&\geq c\\
     y&\geq 0
    \end{align*}
   ausgeschrieben:
    \begin{align*}
   \min\ &5000y_1+11000y_2+8000y_3 &&\text{unter den Nebenbedingungen}\\
    &2y_1+4y_2+3y_3\geq 5\\
    &3y_1+1y_2+4y_3\geq 4\\
    &1y_1+2y_2+2y_3\geq 3\\
    &y_1,y_2,y_3\geq 0
    \end{align*}
   \item
    \begin{itemize}
    \item Erster Versuch in $P$:\\
   Wir nehmen möglichst viel von Müsli A, da dieses den meisten Gewinn bringt.
   Ein zulässiger Vektor wäre $x^1:=\vect1{2500\\0\\0}$. Gewinn wäre $c^Tx^1=12500$.
   \item Erster Versuch in $D$:\\
   Wir wissen bereits, dass $12500$ eine untere Schranke ist, um möglichst nah heranzukommen muss $y_2$ möglichst klein sein. Hier sieht man nun, dass $y^1:=\vect1{1\\0\\1}$ ein zulässiger Vektor ist. Zielfunktion: $b^Ty^1=13000$.\\
   Wir wissen also nun: Das Optimum liegt in $[12500,13000]$.
   \item Zweiter Versuch in $P$:\\
   Da Nüsse knapp sind und Müsli B viel davon verbraucht, versuchen wir $x_1$ groß zu lassen, und ein wenig Müsli C dazu zu nehmen. Da auch Rosinen knapp sind versuchen wir $y_1$ und $y_3$ möglichst genau zu treffen. Wenn wir das Gleichungssystem
   $$\vect2{2&1\\3&2}\vect1{x_1\\x_3}=\vect1{5000\\8000}$$
   lösen, so erhalten wir $x^2_1=2000,x^2_3=1000$. Da $(2000,0,1000)^T$ zulässig ist versuchen wir diesen. Gewinn: $c^Tx^2=13000$. Nun wir wissen wir wegen $b^Ty^1=13000$, dass dies optimal ist.
   \end{itemize} 
 \end{enumerate}
\section*{Aufgabe 6}
 \begin{enumerate}
  \item 
    \begin{itemize}
     \item $\{x\in\R^n|\forall\, 1\leq i\leq n:x_i\leq1\wedge-x_i\leq 1\}$
     \item Setze $I:=\{1,\dots,n\}.\\
           \left\{x\in\R^n|\forall\, S\subseteq I:\left(\sum\limits_{i\in S}x_i-
           \sum\limits_{i\in I\setminus S}x_i\right)\leq 1\right\}$
    \end{itemize}
  \item Setze $c':=\vect1{c\\c_0},d':=\vect1{d\\d_0}$, dann kann man das Problem zunächst umschreiben für $x\in \K^{n+1}$ zu:
  \begin{align*}
  \min\max\ &\{c'^Tx',d'^Tx'\}\\
            &\underbrace{\vect[r]2{A&0\\{\bf 0}&1\\{\bf 0}&-1}}_{=:A'}x'
            \geq \underbrace{\vect[r]1{b\\1\\-1}}_{=:b'}
  \end{align*}
  Mit der erweiterten Ungleichung wird sichergestellt, dass $x'_{n+1}=1$ ist, und somit die ,,Zielfunktionen"' wieder die gleichen sind.\\
  Als nächstes nutzen wir die Gleichheit $\max\{a,b\}=\frac{a+b+|a-b|}2$ und erhalten:
  \begin{align*}
      \min\ &((c'+d')^Tx'+|(c'-d')^Tx'|)\\
            &A'x'\geq b'
  \end{align*}
 \end{enumerate}
\section*{Aufgabe 7}
Gegeben sei ein Graph $G = (V,E)$ mit Kantengewichten $c_e \ge 0$ für alle $e \in E$. Wir konstruieren einen Graphen $G'$ ähnlich wie im Skript S.45 und nehmen an, dass $V = \{v_1,...,v_n\}$ gilt:
\begin{enumerate}
	\item[1.] Die Graphen $G_1 = (U,E_1)$ mit $U := \{u_1,...,u_n\}$ und $G_2 = (W,E_2)$ mit $W := \{w_1,...,w_n\}$ seien knotendisjunkte isomorphe Bilder von $G$, so dass die Abbildungen $v_1 \rightarrow u_i $ und $v_i \rightarrow w_i$, $i = 1,...,n$ Isomorphismen sind.
	\item[2.] Aus $G_2$ entfernen wir das Bild des Knoten $u$, dann verbinden wir die übrigen Knoten $w_i \in W - \{v\}$ mit ihren isomorphen  Bildern $u_i \in U - \{v\}$ durch eine Kante $u_iw_i$ mit dem Gewicht $c(u_i,w_i)=0$. Die Kanten von $G_1$ und $G_2 - {u}$ enthalten das Gewicht ihrer Urbildkanten.
	\item[3.] $G'$ entsteht durch die Vereinigung von $G_1$ und $G_2 - \{u\}$ unter Hinzufügung der Kanten $u_iw_i$.
	\item[4.] Zusätzlich fügen wir $G'$ einen Knoten $v'$ hinzu, der mit unserem Zielknoten v verbunden ist.  
\end{enumerate}

Beweisidee:
- v' soll die hinzugefügten Knoten auf eine gerade Anzahl bringen, sodass ein perfektes Matching möglich ist
- v' muss mit v matchen, sodass es für kein anderes Matching mehr zur Verfügung steht
- Im Graphen $G_2$ können wir den Weg dann von v abhängig machen, anders habe ich keinen Weg gefunden sicherzustellen, dass v irgendwie mit im Matching einbezogen ist
- Matchings zwischen zwei Knoten mit verschiedenem Index stellen dann den Pfad zwischen den Knoten im Originalgraphen dar
- gleicher Index bedeutet kein Pfad
- Eventuell: Brauchen wir die ganzen 0 Kanten wie im Skript gar nicht, nur die Kante zu v' bin mir aber nicht sicher


\section*{Aufgabe 8}

\end{document}
