\documentclass[a4paper,10pt,german]{scrartcl}
\usepackage[T1]{fontenc}
\usepackage[utf8]{inputenc}
\usepackage[ngerman]{babel}
\usepackage{a4wide}
\usepackage{dsfont,alltt}
\usepackage{amsthm,amssymb,amsmath,ifthen,stmaryrd}
\usepackage{tikz}
\usetikzlibrary{arrows}
\newcommand{\komp}[1]{\ensuremath{\overline{#1}}}
\newcommand{\gdw}{\ensuremath{\Leftrightarrow}}
\newcommand{\dann}{\ensuremath{\Rightarrow}}
\newcommand{\K}{\ensuremath{\mathbb{K}}}
\newcommand{\N}{\ensuremath{\mathbb{N}}}
\newcommand{\R}{\ensuremath{\mathbb{R}}}
\newcommand{\one}{\ensuremath{{\mathds{1}}}}
\newcommand{\NP}{\ensuremath{\mathcal{NP\ }}}
\newcommand{\Poly}{\ensuremath{\mathcal{P\ }}}

\newcommand{\ad}[2][k]{\ensuremath{a^{(#1)}_{#2}}}
\newcommand{\aij}[1][k]{\ad[#1]{i,j}}

\newcounter{ctra}
\newcommand{\hilfsstring}{}                             
% \vect[ausrichtung]{n}{text} erzeugt eine Matrix mit runden Klammern der Breite n
% text muss dann & und \\ enthalten, z.B. 11&12&13\\21&22&23
% ausrichtung ist optional und richtet alle Spalten aus, default ist c
\newcommand{\vect}[3][c]{\setcounter{ctra}{0}\renewcommand{\hilfsstring}{} \whiledo{\value{ctra}<#2}{\edef\hilfsstring{#1\hilfsstring}\stepcounter{ctra}}
\ensuremath{\left(\begin{array}{\hilfsstring}#3\end{array}\right)}}
\DeclareMathOperator*{\cone}{cone}
\DeclareMathOperator*{\conv}{conv}
\DeclareMathOperator*{\aff}{aff}

\newenvironment{ggbsp}{\begin{proof}[Gegenbeispiel]}{ \end{proof}}

\renewcommand{\labelenumi}{\alph{enumi}\,)}
\newcommand{\ra}{\Rightarrow}
\renewcommand{\thefootnote}{\fnsymbol{footnote}}

%opening
\title{Einführung in die lineare und kombinatorische Optimierung\\
Serie 3}
\author{Maurice Althoff (FU 4745454)\\Michael R. Jung (HU 502133)\\Felix Völker (TU 331834)}

\begin{document}

\maketitle

\section*{Aufgabe 9}

\section*{Aufgabe 10}
Sei $\mathcal{Z}=\{Z_i|1\leq i\leq m\}$ eine Familie von Klauseln über Variablen aus $X=\{x_1,\dots,x_n\}$.\\
Folgender Algorithmus \verb Alg  findet eine Belegung \verb f  von $X$, die mindestens die Hälfte der Klauseln erfüllt:
\begin{alltt}
Alg:
FOR i=1 TO n:
  BEGIN
    A:=\{Z \(\in\mathcal{Z}\) |x_i \(\in\) Z\}
    B:=\{Z \(\in\mathcal{Z}\) |\(\neg\)x_i \(\in\) Z\}
    IF |A|<|B| THEN f(x_i)=falsch ELSE f(x_i)=wahr
    \(\mathcal{Z}\)=\(\mathcal{Z}\)\(\setminus\)(A\(\cup\)B)
  END
\end{alltt}
\begin{itemize}
 \item Analysieren wir zunächst die Laufzeit:\\
       Eingabegröße ist ungefähr $\sum\limits_{i=1}^m \langle Z_i\rangle =:g$.
       In jedem Schritt müssen wir zur Berechnung von \verb A  und \verb B  maximal alle Klauseln, deren Kodierung insgesamt in der Größe durch $g$ beschränkt ist, durchsuchen. Die entsprechenden Indizes
       in \verb A  und \verb B  zu speichern (Speicherbedarf als Bitmaske: $m$), reicht völlig aus, des Weiteren können wir bereits beim Scan die entsprechenden Klauseln löschen. Also ist die Laufzeit jedes Schleifendurchlaufs
       durch $g+2m$ beschränkt. Da wir insgesamt $n$ Durchläufe haben, ist die ganze Laufzeit durch $n(g+2m)$ beschränkt. Es gilt: $n\leq g,\ m\leq g$.\\
       Folglich haben wir eine Laufzeitschranke von $g(g+2g)\in \mathcal{O}(g^2)$.
 \item Nun zur Korrektheit:\\
       Am Ende ist $\mathcal{Z}=\varnothing$, denn jede Klausel wird entfernt, wenn die Variable mit dem kleinsten vorkommenden Index betrachtet wird. Jedes mal wenn Klauseln entfernt werden, 
       wird durch die definierte Belegung mindestens die Hälfte dieser erfüllt. Da am Ende aber alle Klauseln entfernt werden, wird also mindestens die Hälfte aller Klauseln erfüllt.\hfill $\checkmark$
\end{itemize}


\section*{Aufgabe 11}
Sei $D=(V,A)$ ein Digraph mit $|V|=n$ und Adjazenzmatrix $A\in\R^{n\times n}$. Im folgenden gehen wir o.B.d.A. davon aus, dass $V=\{1,2,\dots,n\}$ gilt. Außerdem sei $A^k=:\left(\aij\right)_{1\leq i,j\leq n}$.\\
Zeigen zunächst: \begin{itemize}
                 \item[] In \aij steht die Anzahl aller verschiedenen Ketten der Länge genau $k$ zwischen den Knoten $i$ und $j$.
                 \end{itemize}
\begin{proof}
  Beweis per Induktion über $k$.\\
  {\bf Induktionsanfang}\\
  Für $k=1$ entspricht dies genau der Definition der Adjazenzmatrix.\\
  {\bf Induktionsvoraussetzung}\\
  In \aij steht die Anzahl der verschiedenen Ketten der Länge genau $k$ zwischen den Knoten $i$ und $j$.\\
  {\bf Induktionsbehauptung}\\
  In \aij[k+1] steht die Anzahl der verschiedenen Ketten der Länge $k+1$ zwischen den Knoten $i$ und $j$.\\
  {\bf Induktionsschritt}\\
  Betrachte \aij[k+1].\\
  Nach Definition der Matrixmultiplikation ist $A^{k+1}=A^kA$ und somit $\aij[k+1]=\sum\limits_{p=1}^n\ad{i,p}\cdot \ad1{p,j}$.\\
  Nach Induktionsvoraussetzung ist \ad{i,p} die Anzahl der verschiedenen Ketten der Länge $k$ von $i$ nach $p$. Dies wird nun mit der Anzahl der Kanten zwischen $p$ und $j$, nämlich \ad[1]{p,j}, multipliziert, und wir erhalten die Anzahl der verschiedenen Ketten zwischen $i$ und $j$ der Länge $k+1$ mit vorletztem Knoten $p$. Wenn wir dies nun über alle möglichen vorletzten Knoten summieren, erhalten wir genau das gewünschte.
\end{proof}
Definiere $\mathcal{A}:=\sum_{p=1}^nA^p$. In $\mathcal{A}_{i,j}$ steht also folglich die Anzahl der verschiedenen Ketten von $i$ nach $j$ der Länge $\leq n$. Ist $j$ von $i$ aus erreichbar, so muss dieser Eintrag verschieden 0 sein, da ein Weg auch eine Kette ist und ein $i$-$j$-Weg höchstens Länge $n$ (für $i\neq j$ sogar nur Länge $n-1$) haben kann.
SCHLUSS FEHLT NOCH.
\section*{Aufgabe 12}
Dass die genannten Probleme in \NP sind, sieht man leicht:
\begin{enumerate}
 \item Das Zertifikat ist eine Permutation $\pi$ der Knoten\footnote{Für einen Graphen $G=(V,E)$ oder einen Digraphen $D=(V,A)$ ist eine Permutation der Knoten eine bijektive Abbildung $\pi:\{1,2,\dots,|V|\}\to V$.}. Hier muss man nun prüfen, ob \\
 $(\pi(1),\dots,\pi(|V|),\pi(1))$ ein Hamiltonkreis ist und, falls ja, ob die Summe der Kantengewichte auf diesem Kreis $K$ nicht übersteigt.
 \item Das Zertifikat ist eine Permutation $\pi$ der Knoten\footnotemark[1]. Hier muss man nun prüfen, ob \\
 $[\pi(1),\dots,\pi(|V|)]$ ein Hamiltonweg und $\pi(1)=u\wedge \pi(|V|)=v$ ist.
 \item Das Zertifikat ist eine Permutation $\pi$ der Knoten\footnotemark[1]. Hier muss man nun prüfen, ob \\
 $[\pi(1),\dots,\pi(|V|)]$ ein Hamiltonweg ist.
 \item Im gerichteten Fall funktionieren die gleichen Zertifikate, hier muss man stattdessen überprüfen ob es die gerichteten Äquivalente sind.
\end{enumerate}
Reduktionen:
\begin{enumerate}
 \item Gegeben sei ein Graph $G=(V,E)$. Wir definieren uns nun noch eine Kantenbewertung $c:E\to \N$ mit $c(e)=1\ \forall\,e\in E$.\\
 Nun gibt es genau dann einen Hamiltonkreis in $G$, wenn es eine Tour in $G$ der Länge $\leq |V|$ gibt.
 \begin{proof}
   In $G$ existiert ein Hamiltonkreis.\\
   \gdw\ Es existiert eine Permutation $\pi$ der Knoten mit $e_i:=\{\pi(i),\pi(i+1)\}\in E\ \forall 1\leq i< |V|$ und $e_{|V|}:=\{\pi(|V|),\pi(1)\}\in E$.\\
   \gdw\ Es existiert eine Permutation $\pi$ der Knoten mit $e_i:=\{\pi(i),\pi(i+1)\}\in E\ \forall 1\leq i< |V|$ und $e_{|V|}:=\{\pi(|V|),\pi(1)\}\in E$ und es gilt $\sum\limits_{i=1}^{|V|}c(e_i)=n$.\\
   \gdw\ Es existiert eine Tour in $G$ mit Gewicht $=|V|$.\\
   \gdw\ Es existiert eine Tour in $G$ mit Gewicht $\leq|V|$.\\
   Die letzte Äquivalenz folgt, da per Konstruktion jede Tour (die ja alle genau $|V|$ Kanten haben) genau Gewicht $|V|$ hat.
 \end{proof}
 \item Gegeben sei ein Graph $G=(V,E)$. Sei $u\in V$ beliebig.\\
 Konstruiere $G'=(V':=V\cup\{u'\},E':=E\cup\{\{v,u'\}|\{v,u\}\in E\})$.\\
 Nun gibt es genau dann einen Hamiltonkreis in $G$, wenn es einen $[u,u']$-Hamiltonweg in $G'$ gibt.
 \begin{proof}
   In $G$ existiert ein Hamiltonkreis.\\
   \gdw\ Es ex. eine Permutation $\pi$ von $V'$ mit $\pi(1)=u,\ \pi(|V'|)=u',\\ \{\pi(i),\pi(i+1)\}\in E\ \forall 1\leq i< |V|$ und $\{\pi(|V|),\pi(1)=u\}\in E$.\\
   (Dies gilt, da der ``Startknoten'' beliebig gewählt werden kann.)\\
   \gdw\ Es ex. eine Permutation $\pi$ von $V'$ mit $\pi(1)=u,\ \pi(|V'|)=u'$ und \\$\{\pi(i),\pi(i+1)\}\in E'\ \forall 1\leq i< |V'|$. (Für $i=|V'|-1=|V|:\{\pi(|V|),u'\}\in E'$.)\\
   (Dies gilt, da $u$ und $u'$ die gleiche Nachbarschaft besitzen.)\\
   \gdw\ Es existiert ein $[u,u']$-Hamiltonweg  in $G'$.
 \end{proof}
 \item Da wir nun bereits wissen, dass das $[u,v]$-Hamiltonweg-Problem \NP-vollständig ist werden wir dieses auf das allgemeine Hamiltonweg-Problem reduzieren.\\
 Sei $G=(V,E)$ ein Graph und seien $u,v\in V$ zwei ausgezeichnete Knoten. $G$ enthält genau dann einen $[u,v]$-Hamiltonweg, wenn $G'=(V':=V\cup\{u',v'\},E':=E\cup\{\{u',u\},\{v,v'\}\})$ einen Hamiltonweg besitzt.
 \begin{proof}~
   \begin{itemize}
    \item[``$\ra$''] Sei $|V|=n$ und sei $[u=v_1,v_2,\dots,v_n=v]$ ein $[u,v]$-Hamiltonweg in $G$, dann ist $[u',v_1,v_2,\dots,v_n,v']$ ein Hamiltonweg in $G'$.
    \item[``$\Leftarrow$''] Sei $[x_0,x_1,\dots,x_{n+1}]$ ein Hamiltonweg in $G'$. Da $\deg(u')=1$ muss gelten, dass $u'\in\{x_0,x_{n+1}\}$, denn alle anderen Knoten müssen ja mindestens Grad 2 haben. Da auch $\deg(v')=1$, gilt:
    $v'\in \{x_0,x_{n+1}\}\setminus \{u'\}$. Also existiert ein $[u',v']$-Hamiltonweg oder ein $[v',u']$-Hamiltonweg, im ungerichteten Fall also beide. Sei nun o.B.d.A. $x_0=u'$ und $x_{n+1}=v'$.
    Da $u'$ nur einen Nachbarn, nämlich $u$, und $v'$ nur einen Nachbarn, nämlich $v$ hat, muss gelten: $x_1=u\wedge x_n=v$. Also ist $[x_1,x_2,\dots,x_n]$ ein $[u,v]$-Hamiltonweg in $G$.
   \end{itemize}
  \end{proof}
  \item Die Konstruktioen und Beweise funktionieren genauso, dabei haben wir darauf geachtet, dass man bei hinzugefügten Bögen die geschweiften Klammern nur
  durch runde zu ersetzten braucht, d.h. die richtige Orientierung ist bereits beachtet. In a) verändert sich also tatsächlich nichts, außer dass $c$ nun eine Bogenbewertung ist.
  In b) bekommt $u'$ nur die eingehenden Kanten von $u$ und in c) ist durch die Orientierung bereits sichergestellt, dass ein gerichteter Hamiltonweg bereits ein $[u',v']$-Hamiltonweg ist.
\end{enumerate}
\end{document}
