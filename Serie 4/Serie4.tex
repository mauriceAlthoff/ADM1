\documentclass[a4paper,10pt,german]{scrartcl}
\usepackage[T1]{fontenc}
\usepackage[utf8]{inputenc}
\usepackage[ngerman]{babel}
\usepackage{a4wide}
\usepackage{dsfont,alltt}
\usepackage{amsthm,amssymb,amsmath,ifthen,stmaryrd}
\usepackage{tikz}
\usetikzlibrary{arrows}

%opening
\title{Einführung in die lineare und kombinatorische Optimierung\\
Serie 4}
\author{Maurice Althoff (FU 4745454)\\Michael R. Jung (HU 502133)\\Felix Völker (TU 331834)}

\begin{document}

\maketitle

\section{Aufgabe 15}

Sei $D = (V,A)$ ein Digraph mit $n \ge 2$ Knoten.\\
Zu zeigen: $(1) \Rightarrow (2)$\\
Wir nehmen an, dass D eine Arboreszenz ist(A1) und zerlegen die Aussage in zwei Teilbeweise.\\
\\
Zu zeigen(Z1): D hat $n-1$ Bögen.\\ 
Wir beweisen die Aussage per Widerspruch und nehmen an, dass D keine $n-1$ Bögen besitzt(A2).\\
Somit gibt es zwei Fälle:\\
\\
Fall 1: $|A| < n - 1$\\
\\
Da $D$ genau $n$ Knoten besitzt und es nur maximal $n - 2$ Bögen geben kann, ex. ein Knoten $u \in V$, der auf keinem Weg w mit maximaler Länge enthalten ist. Dies gilt, da ein Weg über alle Knoten mindestens $n - 1$ Kanten hätte. Daraus folgt, dass $\{uv,vu\} \cap A = \emptyset$ für alle $v \in V / \{u\}$ und somit kann $D$ kein zusammenhängender Graph bzw. ein Baum, sowie keine Arboreszenz sein. \textbf{Widerspruch!}\\
\\
Fall 2: $|A| > n - 1$\\
\\
Da $D$ genau $n$ Knoten besitzt und es mindestens $n$ Bögen gibt. Nehmen wir an, dass aus allen Knoten eine Kante ausgehen, dann  für einen beliebigen Knoten $v \in V$ gelten, dass n
\\
Zu zeigen(Z2): D ist quasi-stark zusammenhängend.\\

Zu ze
\\
Zu zeigen: $(2) \Rightarrow (3)$\\
Annahme(A1): $D$ hat $n-1$ Bögen und ist quasi-stark zusammenhängend. \\
Zu zeigen: D enthält einen Knoten $r$, so dass es in $D$ für jeden Knoten $v$ genau einen gerichteten $r,v$-Weg gibt.
Da A1 gilt, folgt dass für jedes Paar aus Knoten $u,v \in V$ ein Knoten w ex. , so dass es von $w$ einen gerichteten Weg zu $u$ und einen gerichteten Weg zu $v$ gibt. Setzen wir nun $r = w$, so enthält $D$ einen Knoten $r$, so dass es einen gerichteten $(r,u)$-Weg und einen gerichteten $(r,v)$-Weg in $D$ gibt. Da $u,v$ ein beliebes Paar aus Knoten ist, folgt dass für jeden Knoten $v'$ ein gerichteter $(r,v')$- Weg existiert.\\
Zusätzlich gilt, dass $r$ eine Kante, bzw. einen gerichteten Weg zu allen anderen $n-1$ Knoten besitzt. Das heißt es muss min. $n-1$ Kanten geben, die $r$ mit den anderen Knoten $v$ direkt oder über einen Weg verbinden, da sonst kein $(r,v)$-Weg existieren würde. Da per A1 gilt, dass $D$ genau $n - 1$ Knoten hat, kann nur genau ein gerichteter Weg existieren. DIes ist der Fall, da für jeden neuen Weg eine zusätzlich Kante benötigt werden würde, also mindestens $n$ viele Kanten. Dies wäre jedoch ein Widerspruch zur Annahme A1.
Somit gilt die Aussage.\\
\\
Zu zeigen: $(3) \Rightarrow (4)$\\
\\
Annahme(A1): $D$enthält einen Knoten $r$, so dass es in $D$ für jeden anderen Knoten $v$ genau ein gerichteten $(r,v)$-Weg gibt.\\
Zu zeigen(Z1): $D$ ist quasi-stark zusammenhängend.\\
Da A1 gilt, gibt es auch für ein beliebiges Paar von Knoten $u,v \in V$ einen gerichteten $(r,u)$-Weg und einen gerichteten $(r,v)$-Weg. Somit ist $D$ nach Definition quasi-zusammenhängend. 
Zu zeigen(Z2): $D$ besitzt einen Knoten $r$ mit $\delta^-(r) = 0$ und erfüllt $\delta^-(v) = 1$ für alle $v \in V \\ \{r\}$.\\
Da A1 gilt, muss $r$ der Knoten, der zu allen anderen Knoten $v \in V$ einen gerichteten $(r,v)$-Weg besitzt, mit $\delta^-(r) = 0$ sein. Hätte $r$ nämlich einen Innengrad größer $0$, so gäbe es folgendermaßen einen Kreis in $D$(denn es ex. ein gerichteter Weg von $r$ zu allen anderen Knoten $v$). Da deswegen auch mehr als ein gerichteten Weg von $r$ zu den anderen Knoten $v$ existieren kann, indem man mehrmals über den Knoten $r$ läuft, entsteht ein Widerspruch zur Annahme A1.\\
Zusätzlich müssen alle anderen Knoten $v \in V \\ \{r\}$ den Innengrad 1 besitzen, da es nur genau einen gerichteten $(r,v)$-Weg gibt. Hätte ein Knoten $v' \in V \\ \{r\}$ einen Innengrad von 0, so gäbe es keinen gerichteten $(r,v')$-Weg. Hätte $v'$ einen Innengrad größer 1, so gäbe es zwei verschiedene $(r,v')$-Wege, da $r$ einen gerichteten Weg zu allen anderen Knoten besitzt. Beide Fälle stehen im Widerspruch zur Annahme.
Somit gilt die Aussage.\\
\\
Zu zeigen: $(4) \Rightarrow (5)$\\
Annahme(A1): $D$ ist quasi-stark zusammenhängend, besitzt einen Knoten $r$ mit $\delta^-(r) = 0$ und erfüllt $\delta^-(v) = 1$ für alle $v \in V \\ \{r\}$.\\
Zu zeigen: D enthält keinen Kreis, einen Knoten $r$ mit $\delta^-(r) = 0$ und erfüllt $\delta^-(v) = 1$ für alle $v \in V \\ \{r\}$.\\
\\
Da A1 gilt, gibt es für zwei bel. Paare von Knoten $u,v \in V$, einen Knoten $r$, so dass es einen gerichteten $(r,u)$-Weg und einen gerichteten $(r,v)$- Weg gibt. $r$ muss dabei den Innengrad 0 besitzen, da sonst ein bel. Knoten $v' \in V$ den Innengrad 0 hätte, aber dann kein gerichteter $(r,v')$-Weg existieren würde. Dies würde der Annahme widersprechen. Somit müssen alle Knoten $v$ den Innengrad 1 besitzen. 
Zusätzlich gilt, dass $r$ kein Koten einer Kante in einem Kreis sein kann, da sonst $r$ einen Innengrad größer 0 hätte. Für alle anderen Knoten $v$ gilt, dass es einen gerichteten $(r,v)$-Weg gibt und somit diese Knoten nicht Teil einer Kanten in einem Kreis wären. Dies ist der Fall da diese Knoten sonst einen Innengrad größer 1 hätten(da der gerichtete $(r,v)$-Weg bereits eine eingehende Kante in $v$ erfordert). Somit kann kein Knoten Teil einer Kante in einem Kreis sein und folgendermaßen kann kein Kreis existieren. Somit gilt die Aussage.\\ 
\\
Zu zeigen: $(5) \Rightarrow (1)$\\
\\
Da $D$ genau $n$ Knoten besitzt und damit $n - 1$ viele Knoten enthält, die einen Innengrad von 1 besitzen, muss $D$ genau $n - 1$ Kanten besitzen. Hätte $D$ mehr Kanten, so würde $deg^-(r) > 0$ oder $deg^-(v) > 1$ für einen Knoten $v \in V \\ \{r\}$. Hätte $D$ weniger Kanten als $n - 1$ Kanten, so würde $deg^-(v)=0$ für einen Knoten $v \in V \\ \{r\}$ gelten. Da $D$ kreisfrei ist muss es somit einen Weg der Länge $n - 1$ geben, der über alle Knoten $v' \in V$ läuft(da sonst ein Knoten doppelt im Weg vorkommen und $D$ somit einen Kreis enthalten würde). Daraus folgt, dass D zusammenhängend sein muss. Aus dieser Folgerung, der Annahme, dass $D$ kreisfrei ist und für alle Knoten $v' \in V$ $deg^-(v') \le 1$ gilt, folgt per Definition, dass $D$ eine Arboreszenz ist. 

\section{Aufgabe 16}

Eingabe: ein Graph $ G=(V,E), c \in E $ mit Kantengewichten $c(W) \forall e \in E$ \\
Ausgabe: Wald $W \subseteq E$ mit max Gewicht $c(W)$ \\
\begin{enumerate}\itemsep1pt \parskip0pt \parsep0pt
\item (Sortieren): Ist $k$ die Anzahl der Kanten von G mit positivem Gewicht, so numeriere diese $k$ Kanten, so dass gilt $ c(e_1) \geq c(e_2) \geq \ldots \geq c(e_k) > 0$.\\
\item Setze $W := \emptyset $ .\\
\item FOR $i=1$ TO $k$ DO: \\
\-\hspace{1cm} Falls $W \cup \{ e_i\}$ keinen Kreis enth\"alt, setze $W:=W \cup \{ e_i\} $ \\
\item Gib $W$ aus.
\end{enumerate}

\textbf{Induktionsannahme}: \\
$W_{i-1}$ ist ein maximaler Wald, der die ersten $i-1$ vom Greedy-Max Algorithmus bestimmte Kanten $e_1, ..., e_{i-1}$ enth\"alt.\\
\textbf{Induktionsschritt}: $i-1 \rightarrow i$: \\
Zu seigen, es gibt einen maximalen Wald $W_i$, der die vom Algorithmus ausgew\"ahlten Kanten $e_j \forall j\geq i$ enth\"alt.\\
Der Algorithmus w\"ahlt die im i-ten Schritt die Kante $e_i$ aus, f\"ur diese Kante muss gelten: \\
$c(e_1) \geq c(e_K) \forall \notin W_{i-1}$, so dass $W_{i-1} \cup \{ e_K\}$ keinen Kreis enth\"alt.\\
Da $W_{i-1}$ einen Wald ist, insbesondere $\forall e_K \in W_{i-1} \setminus \{ e_1, ..., e_{i-1}\}$, d.h. f\"ur alle Kanten in $W_{i-1}$, die der Greedy Algorithmus noch nicht gew\"ahlt hat.\\
F\"uge nun diese Kante $e_i$ zu $W_{i-1}$ hinzu. Dann entsteht in $W_{i-1}$ ein Kreis, da $W_{i-1}$ bereits ein maximaler Wald war und durch hinzuf\"ugen einer Kante genau ein Kreis entsteht.\\
Entfernte aus diesem Kreis die Kante $K$, wobei $k \neq e_j \forall j \geq i$ ist. Diese Kante existiert in $W_{i-1}$, da der Greedy Max-Algorithmus sonst einen Kreis fabriziert h\"atte.\\
D.h. $W_i := ( W_{i-1} \setminus \{ k\} ) \cup \{ e_i\} $ ist ein Wald, der $e_j \forall j \geq i$ enth\"alt und ausserdem maximal ist, da $c(e_i) \geq c(k)$.
\end{document}
