\documentclass[a4paper,10pt,german]{scrartcl}
\usepackage[T1]{fontenc}
\usepackage[utf8]{inputenc}
\usepackage[ngerman]{babel}
\usepackage{a4wide}
\usepackage{dsfont,alltt}
\usepackage{amsthm,amssymb,amsmath,ifthen,stmaryrd}
\usepackage{tikz}
\usetikzlibrary{arrows}

%opening
\title{Einführung in die lineare und kombinatorische Optimierung\\
Serie 4}
\author{Maurice Althoff (FU 4745454)\\Michael R. Jung (HU 502133)\\Felix Völker (TU 331834)}

\begin{document}

\maketitle

\section{Aufgabe 16}

Eingabe: ein Graph $ G=(V,E), c \in E $ mit Kantengewichten $c(W) \forall e \in E$ \\
Ausgabe: Wald $W \subseteq E$ mit max Gewicht $c(W)$ \\
\begin{enumerate}\itemsep1pt \parskip0pt \parsep0pt
\item (Sortieren): Ist $k$ die Anzahl der Kanten von G mit positivem Gewicht, so numeriere diese $k$ Kanten, so dass gilt $ c(e_1) \geq c(e_2) \geq \ldots \geq c(e_k) > 0$.\\
\item Setze $W := \emptyset $ .\\
\item FOR $i=1$ TO $k$ DO: \\
\-\hspace{1cm} Falls $W \cup \{ e_i\}$ keinen Kreis enth\"alt, setze $W:=W \cup \{ e_i\} $ \\
\item Gib $W$ aus.
\end{enumerate}

\textbf{Induktionsannahme}: \\
$W_{i-1}$ ist ein maximaler Wald, der die ersten $i-1$ vom Greedy-Max Algorithmus bestimmte Kanten $e_1, ..., e_{i-1}$ enth\"alt.\\
\textbf{Induktionsschritt}: $i-1 \rightarrow i$: \\
Zu seigen, es gibt einen maximalen Wald $W_i$, der die vom Algorithmus ausgew\"ahlten Kanten $e_j \forall j\geq i$ enth\"alt.\\
Der Algorithmus w\"ahlt die im i-ten Schritt die Kante $e_i$ aus, f\"ur diese Kante muss gelten: \\
$c(e_1) \geq c(e_K) \forall \notin W_{i-1}$, so dass $W_{i-1} \cup \{ e_K\}$ keinen Kreis enth\"alt.\\
Da $W_{i-1}$ einen Wald ist, insbesondere $\forall e_K \in W_{i-1} \setminus \{ e_1, ..., e_{i-1}\}$, d.h. f\"ur alle Kanten in $W_{i-1}$, die der Greedy Algorithmus noch nicht gew\"ahlt hat.\\
F\"uge nun diese Kante $e_i$ zu $W_{i-1}$ hinzu. Dann entsteht in $W_{i-1}$ ein Kreis, da $W_{i-1}$ bereits ein maximaler Wald war und durch hinzuf\"ugen einer Kante genau ein Kreis entsteht.\\
Entfernte aus diesem Kreis die Kante $K$, wobei $k \neq e_j \forall j \geq i$ ist. Diese Kante existiert in $W_{i-1}$, da der Greedy Max-Algorithmus sonst einen Kreis fabriziert h\"atte.\\
D.h. $W_i := ( W_{i-1} \setminus \{ k\} ) \cup \{ e_i\} $ ist ein Wald, der $e_j \forall j \geq i$ enth\"alt und ausserdem maximal ist, da $c(e_i) \geq c(k)$.
\end{document}
