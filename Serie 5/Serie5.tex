\documentclass[a4paper,10pt,german]{scrartcl}
\usepackage{mathtools}
\usepackage[T1]{fontenc}
\usepackage[utf8]{inputenc}
\usepackage[ngerman]{babel}
\usepackage{a4wide}
\usepackage{dsfont,alltt}
\usepackage{amsthm,amssymb,amsmath,ifthen,stmaryrd}
\usepackage{tikz}
\usetikzlibrary{arrows}
\usepackage{listings}

%opening
\title{Einführung in die lineare und kombinatorische Optimierung\\
Serie 5}
\author{Sven-Maurice Althoff (FU 4745454)\\Michael R. Jung (HU 502133)\\Felix Völker (TU 331834)}

\begin{document}

\maketitle

\section{Aufgabe 18}

Wir modellieren die Stunden als Knoten und die Kosten als Kanten, um einen zusammenh\"angenden Graphen zu erhalten m\"ussen wir Kanten mit keinen Kosten einf\"uhgen die von jeder Stunde auf die vorherige Stunde zeigen. Dadurch erhalten wir diesen Digraph:

\begin{tikzpicture}[->,>=stealth',shorten >=1pt,auto,node distance=2cm,
  thick,main node/.style={circle,draw,font=\sffamily\Large\bfseries}]

  \node[main node] (9) {9};
  \node[main node] (10) [right of=9] {10};
  \node[main node] (11) [right of=10] {11};
  \node[main node] (12) [right of=11] {12};
  \node[main node] (13) [right of=12] {13};
  \node[main node] (14) [right of=13] {14};
  \node[main node] (15) [right of=14] {15};
  \node[main node] (16) [right of=15] {16};
  \node[main node] (17) [right of=16] {17};

  \path[every node/.style={font=\sffamily\small}]
    (10) edge node {0} (9)
    (10) edge node {0} (9)
    (11) edge node {0} (10)
    (12) edge node {0} (11)
    (13) edge node {0} (12)
    (14) edge node {0} (13)
    (15) edge node {0} (14)
    (16) edge node {0} (15)
    (17) edge node {0} (16)

    (9) edge [bend right] node {30} (13)
    (9) edge [bend left] node {18} (11)
    (12) edge [bend left] node {21} (15)
    (12) edge [bend left] node {38} (17)
    (14) edge [bend right] node {20} (17)
    (13) edge [bend right] node {22} (16)
    (15) edge [bend left] node {9} (17);
\end{tikzpicture}
Die L\"osung des Problems ist der k\"urzeste Pfad von Knoten 9 nach 17.\\
Um den k\"urzesten Weg zu ermitteln wird Dijkstra-Algorithmus aus der Vorlesung verwendet:

\begin{center}
\resizebox{\textwidth}{!}{%
  \begin{tabular}{| c | l | r | c | c | c | c | c | c | c | c | c | c  l |}
    \hline
     Step & $\lnot$Visit                 &      Visit                     & C  & 9              & 10             & 11             & 12             & 13             & 14             & 15             & 16             & 17             &\\ \hline
     0 & $\{ 9,10,11,12,13,14,15,16,17\} $ & $\{ \emptyset \} $           &    & $(\infty , -)$ & $(\infty , -)$ & $(\infty , -)$ & $(\infty , -)$ & $(\infty , -)$ & $(\infty , -)$ & $(\infty , -)$ & $(\infty , -)$ & $(\infty , -)$ & Init\\ \hline
     1 & $\{ 10,11,12,13,14,15,16,17\}   $ & $\{ 9 \} $                   & 9  &                & $(\infty , -)$ & $(18 , 9)$     & $(\infty , -)$ & $(30 , 9)$     & $(\infty , -)$ & $(\infty , -)$ & $(\infty , -)$ & $(\infty , -)$ &\\ \hline
     2 & $\{ 10,12,13,14,15,16,17\}      $ & $\{ 9,11 \} $                & 11 &                & $(18 , 10)$    &                & $(\infty , -)$ & $(30 , 9)$     & $(\infty , -)$ & $(\infty , -)$ & $(\infty , -)$ & $(\infty , -)$ &\\ \hline
     3 & $\{ 12,13,14,15,16,17\}         $ & $\{ 9,11,10 \} $             & 10 &                &                &                & $(\infty , -)$ & $(30 , 9)$     & $(\infty , -)$ & $(\infty , -)$ & $(\infty , -)$ & $(\infty , -)$ & Dead End\\ \hline
     4 & $\{ 12,14,15,16,17\}            $ & $\{ 9,11,10,13 \} $          & 13 &                &                &                & $(30 , 13)$    &                & $(\infty , -)$ & $(\infty , -)$ & $(\infty , -)$ & $(\infty , -)$ &\\ \hline
     5 & $\{ 14,15,16,17\}               $ & $\{ 9,11,10,13,12 \} $       & 12 &                &                &                &                &                & $(\infty , -)$ & $(51 , 12)$    & $(\infty , -)$ & $(68 , 12)   $ &\\ \hline
     6 & $\{ 14,16,17\}                  $ & $\{ 9,11,10,13,12,15 \} $    & 15 &                &                &                &                &                & $(51 , 13)$    &                & $(\infty , -)$ & $(60 , 13)$    &\\ \hline
     7 & $\{ 16,17\}                     $ & $\{ 9,11,10,13,12,15,14 \} $ & 14 &                &                &                &                &                &                &                & $(\infty , -)$ & $(60 , 13)$    & Dead End\\ \hline
  \end{tabular}}
\end{center}

Ergebnis:\\

Die k\"urzeste Strecke ist $\{9-13,12-15,15-17\} $ mit kosten von 60.

\section{Aufgabe 19}
Sei $D = (V,A)$ ein Digraph mit Gewichten $c_a \in \mathbb{R}$, $c_a \ge$ 0 f\"ur jeden Bogen $a \in A$.\\
Sei $s \in V$ und $A_T die Menge der B"ogen in T$.\\
\\
Zu zeigen: $T$ ist ein K\"urzester-Wege-Baum von $s$ $\Leftrightarrow$ $\forall (u,v) \in A \setminus A_T. d_T(s,u) + c(u,v) \ge d_T(s,v)$
Wir beweisen die beiden Implikationen.\\
Zu zeigen(Z1): $T$ ist ein Kürzester-Wege-Baum von $s$ $\Rightarrow$ $\forall (u,v) \in A \setminus A_T. d_T(s,u) + c(u,v) \ge d_T(s,v)$\\
Wir beweisen Z1 per Kontraposition.
\begin{align*}
	&\exists (u,v) \in A \setminus A_T: d_T (s,u) + c(u,v) < d_T (s,v)\\
	\xRightarrow{} &Es\ ex.\ ein\ (s,v)-Weg\ w = (s,...,u,(u,v),v)\ in\ D,\ so\ dass\\
	&w\ ein\ k"urzerer\ Weg\ ist\ als\ der\ (s,v)-Weg\ in\ T\\
	\xRightarrow{} &T\ enth"alt\ nicht\ den\ k"urzesten\ (s,v)-Weg\\
	\xRightarrow{} &T\ ist\ kein\ K"urzester-Wege-Baum\ von\ s\\
\end{align*}
Da wir die Kontraposition gezeigt haben, gilt auch Z1.\\
\\
Zu zeigen: $\forall (u,v) \in A \setminus A_T. d_T(s,u) + c(u,v) \ge d_T(s,v)$ $\Rightarrow$ $T$ ist ein Kürzester-Wege-Baum von $s$\\ 
\begin{align*}
	&\forall (u,v) \in A \setminus A_T. d_T (s,u) + c(u,v) \ge d_T (s,v)\\
	\xRightarrow{} &Alle\ (s,v)-Wege\ w = (s,...,u,(u,v),v)\ in\ D\ haben\ mindestens\\
	&die\ gleiche\ L"ange\ wie\ der\ (s,v)-Weg\ in\ T\\
	\xRightarrow{} &Der\ (s,v)-Weg\ in\ T\ ist\ der\ k"urzeste\ Weg\ in\ D \\
	\xRightarrow{} &T\ ist\ ein\ K"urzester-Wege-Baum\ von\ s\\
\end{align*}
Da wir die beiden Implikationen bewiesen haben, gilt die ursprüngliche Aussage.

\section{Aufgabe 20}
a)\\
\begin{align}
	&c'(v,w) = c(v,w) - f(v,t) + f(w,t) | + f(v,t)\\
	\xRightarrow{} &c'(v,w) + f(v,t) = c(v,w) + f(w,t)\\
	\xRightarrow{Dr-Ungl} &c'(v,w) + f(v,t) \ge f(v,t) | - f(v,t)\\
	\xRightarrow{} &c'(v,w) \ge 0
\end{align}
b) Dieses Verfahren hat Vorteile f\"ur den Dijkstra-Algorithmus, da er so auch f\"ur negeative Gewichtete Digraphen verwendet werden kann.
\end{document}
