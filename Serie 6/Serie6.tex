\documentclass[a4paper,12pt,german]{scrartcl}
%\usepackage{mathtools}
\usepackage{listings}
\usepackage[T1]{fontenc}
\usepackage[utf8]{inputenc}
\usepackage[ngerman]{babel}
\usepackage{a4wide}
\usepackage{dsfont,alltt}
\usepackage{amsthm,amssymb,amsmath,ifthen,stmaryrd}
\usepackage{tikz}
\usetikzlibrary{arrows}
\usepackage{listings}
\renewcommand{\labelenumi}{\alph{enumi})}

%opening
\title{Einführung in die lineare und kombinatorische Optimierung\\
Serie 6}
\author{Sven-Maurice Althoff (FU 4745454)\\Michael R. Jung (HU 502133)\\Felix Völker (TU 331834)}

\begin{document}

\maketitle

\section*{Aufgabe 20}
Zeigen wir zunächst "$\Leftarrow$":\\
Zu prüfen ist, ob der so definierte Fluss $x$ zulässig ist.\\
Per Voraussetzung gilt: $\forall uv\in A:0\leq x_{uv}\leq c_{uv}.\checkmark$\\
Betrachte beliebigen Knoten $u\in V\setminus \{s,t\}$:\\
Dieser liege auf den Wegen $P_{i_1},\dots,P_{i_p}$ und den Kreisen $C_{j_1},\dots C_{j_q}$.
Seien $x_{i_1},\dots,P_{i_x}$ seine Vorgänger und $y_{i_1},\dots,y_{i_p}$ seine Nachfolger auf den Pfaden sowie $v_{j_1},\dots v_{j_q}$ seine Vorgänger und $w_{j_1},\dots w_{j_q}$ seine Nachfolger auf den Kreisen.\\
Dann ist \begin{align*}
\sum\limits_{z\in\delta^-(u)}x_{zu}
&=\sum\limits_{i\in\{i_1,\dots,i_p\}}x_{x_iu}+ \sum\limits_{j\in\{j_1,\dots,j_q\}}x_{v_ju}\\
&=\sum\limits_{i\in\{i_1,\dots,i_p\}}\lambda_i+ \sum\limits_{j\in\{j_1,\dots,j_q\}}\mu_j\\
&=\sum\limits_{i\in\{i_1,\dots,i_p\}}x_{uy_i}+ \sum\limits_{j\in\{j_1,\dots,j_q\}}x_{uw_j}\\
&=\sum\limits_{z\in\delta^+(u)}x_{uz}
\end{align*}
Somit ist der Fluss zulässig.\\
"$\Rightarrow$":\\
Wir geben einen Algorithmus an, der sukzessive die Pfade und Kreise mit entsprechenden $\lambda_i$ resp. $\mu_i$ findet.\\
Wir beginnen mit einer beliebigen Kante $su$ mit $x_su>0$. Hier gehen wir solange  über Kanten mit Fluss gröer 0 weiter, bis wir entweder $t$ erreichen oder einen Kreis schließen. Im ersten Fall wählen wir $\lambda$ als das Minimum des Flusses auf allen beteiligten Kanten, im zweiten Fall wählen wir $\mu$ als Minimum des Flusses auf den am Kreis beteiligten Kanten. Auf diesen (also ganzer Pfad oder gefundenem Kreis) verringern wir den Fluss um $\lambda$ resp. $\mu$ und beginnen den Algorithmus von vorn. Dieser terminiert weil jedes Mal auf mindestens einer Kante der Fluss 0 wird. Sollte nun oder von Beginn an keine von $s$ ausgehende Kante einen Fluss $>0$ haben, so ist der Fluss überall $0$ oder aber zumindest $t$ hat keine eingehende Kante mehr mit Fluss $>0$ die nicht auf einem Kreis liegt, da die berechneten Flüsse zulässig bleiben und $\sum\limits_{v\in\delta^+(s)}x_{sv}-\sum\limits_{v\in\delta^-(s)}x_{vs} =\sum\limits_{v\in\delta^-(t)}x_{vt}-\sum\limits_{v\in\delta^+(t)}x_{tv}$.
Somit besteht wegen des Flusserhalts der aktuelle Fluss nur noch aus einer Vereinigung von Kreisen. Diese finden wir so ähnlich wie bisher, nur dass wir jetzt mit einem beliebigen Knoten $u$ mit $\sum\limits_{v\in\delta^+(su)}x_{uv}>0$ beginnen.
\section*{Aufgabe 21} 

  \begin{enumerate}
    \item Diese Aussage ist falsch. Betrachte folgendes Gegenbeispiel:
    \begin{center}
    \begin{tikzpicture}[every node/.style={circle},node distance=40pt]
    \node[draw] (1) {$s$};
    \node[draw] (2)[right of=1] {$u$} edge[<-]node[above]{1} (1);
    \node[draw] (3)[right of=2] {$v$} edge[->,bend left]node[below]{1} (2) edge[<-,bend right]node[above]{1} (2);
    \node[draw] (4)[right of=3] {$t$} edge[<-]node[above]{1} (3);
    \node[draw] (5)[below of=2] {$x$} edge[<-]node[left]{1} (2);
    \node[draw] (6)[below of=3] {$y$} edge[->]node[right]{1} (3)edge[<-]node[below]{1}(5);
    \end{tikzpicture}    
    \end{center}
    Hier können wir alle Bögen saturieren (d.h. $x_a=c_a\,\forall a\in A$) und erhalten einen maximalen Fluss. Insbesondere ist aber $x_{uv}=x_{vu}=1$.

  \item Diese Aussage ist wahr. Seien ein Netzwerk $((V,A),c,s,t)$ und ein maximaler Fluss $x_a,a\in A$ gegeben. Seien $u,v\in V$ mit $x_{uv}\neq0\neq x_{vu}$. O.B.d.A. sei $x_{uv}\geq x_{vu}$ (sonst vertausche $u$ und $v$). Dann ist auch $x'_{a},a\in A$ ein maximaler Fluss mit $x'_a=x_a\,\forall a\in A\setminus\{(u,v),(v,u)\}$ und $x'_{uv}=x_{uv}-x_{vu},x'_{v,u}=0$.\\
  Man sieht, dass für jeden Knoten $v$ gilt: $\sum\limits_{a\in\delta^+(v)}x'_a-\sum\limits_{a\in\delta^-(v)}x'_a=\sum\limits_{a\in\delta^+(v)}x_a-\sum\limits_{a\in\delta^-(v)}x_a$. Dies gilt, da sich für alle Knoten außer $u$ und $v$ nichts ändert und für diese beiden sich beide Summen um den gleichen Betrag ändern. Außerdem bleibt der Fluss auch maximal, da sich die Bilanz auch für $s$ nicht geändert hat. Dieses Verfahren kann man sooft wiederholen, bis die geforderte Bedingung für alle Knotenpaare erfüllt ist, da zwei solche Veränderungen sich gegenseitig nicht beeinflussen.
  \item Diese Aussage ist falsch. Betrachte folgendes Gegenbeispiel:
    \begin{center}
    \begin{tikzpicture}[every node/.style={circle},node distance=60pt]
    \node[draw] (1) {$s$};
    \node[draw] (2)[above right of=1] {$u$} edge[<-]node[above]{9} (1);
    \node[draw] (3)[right of=2] {$v$} edge[<-]node[above]{5} (2);
    \node[draw] (4)[below right of=3] {$t$} edge[<-]node[right]{13} (3);
    \node[draw] (5)[below right of=1] {$x$} edge[<-]node[below]{11} (1) edge[->,bend left]node[above]{8}(3);
    \node[draw] (6)[right of=5] {$y$} edge[->]node[right]{7} (4)edge[<-]node[below]{3}(5)
    edge[<-,bend left]node[right]{4}(2);
    \end{tikzpicture}
    \end{center}
    Hier sind alle Kapazitäten paarweise verschieden, aber sowohl $B:=\delta^+(\{s\})$ als auch $C:=\delta^+(\{s,u,x\})$ minimale $(s,t)$-Schnitte mit $c(B)=c(C)=20$. Minimal sind sie, denn es existiert ein Fluss $f_a,a\in A$ mit $f_a=c_a\,\forall a\in A$ und Wert 20. Im Übrigen ist hier sogar jeder $(s,t)$-Schnitt minimal.
    
  \item Diese Aussage ist wahr. Sei $B\subseteq$ ein beliebiger $(s,t)$-Schnitt und $c'_a=\lambda c_a\,\forall a\in A$. Dann ist nach der Veränderung $c'(B)=\sum\limits_{a\in B}c'_a=\sum\limits_{a\in B}\lambda c_a=\lambda\sum\limits_{a\in B}c_a$. Nun ist klar, dass ein minimaler Schnitt auch minimal bleibt, da für beliebige $(s,t)$-Schnitte $S_1,S_2$ mit $c(S_1)\leq c(S_2)$ gilt: $c'(S_1)\leq c'(S_1)\iff \lambda c(S_1)\leq \lambda c(S_2)\stackrel{\lambda>0}{\iff}c(S_1)\leq c(S_2)$.
  \item Diese Aussage ist falsch. Betrachte folgendes Gegenbeispiel:
    \begin{center}
    \begin{tikzpicture}[every node/.style={circle},node distance=60pt]
    \node[draw] (1) {$s$};
    \node[draw] (2)[above right of=1] {$u$} edge[<-]node[above left]{1} (1);
    \node[draw] (3)[below right of=2] {$v$} edge[<-]node[above right]{2} (2);
    \node[draw] (4)[right of=3] {$t$} edge[<-]node[above]{3} (3);
    \node[draw] (5)[below right of=1] {$x$} edge[<-]node[below left]{1} (1) edge[->]node[below right]{2}(3);
    \end{tikzpicture}
    \end{center}
    Hier wäre $B:=\delta^+(s)$ ein minimaler $(s,t)$-Schnitt mit $c(B)=2$. Erhöhen wir aber jede Kapazität um 2, so erhalten wir folgendes Netzwerk:
        \begin{center}
    \begin{tikzpicture}[every node/.style={circle},node distance=60pt]
    \node[draw] (1) {$s$};
    \node[draw] (2)[above right of=1] {$u$} edge[<-]node[above left]{3} (1);
    \node[draw] (3)[below right of=2] {$v$} edge[<-]node[above right]{4} (2);
    \node[draw] (4)[right of=3] {$t$} edge[<-]node[above]{5} (3);
    \node[draw] (5)[below right of=1] {$x$} edge[<-]node[below left]{3} (1) edge[->]node[below right]{4}(3);
    \end{tikzpicture}
    \end{center}
    Hier ist nun $\delta^+(s)$ kein minimaler $(s,t)$-Schnitt mehr, denn $c(\delta^+(s))=6$, \linebreak aber $c(\delta^+(\{s,u,x,v\}))=5$.\\
    Das Problem bei dieser Veränderung ist, dass  ein $(s,t)$-Schnitt in der Anzahl der beteiligten Kanten skaliert wird.
  \end{enumerate}  

\section*{Aufgabe 23}
Im folgenden nennen wir ein kardinalitätsmaximales Matching ein \emph{größtes} Matching, und einen maximalen, zulässigen $(s,t)$-Fluss einen \emph{größten} Fluss.\\
Zunächst zeigen wir, dass der Wert eines größten Matchings dem eines größten Flusses entspricht.\\
Sei also ein größtes Matching $M\subseteq E$ gegeben. Betrachte nun die gerichtete Variante $M'=\{(u,v)\in V_1\times V_2|\{u,v\}\in M\}$. Dann ist 
$$x_a=\begin{cases}1 &a\in \{(s,u)|\exists v\in V_2:uv\in M'\}\cup M'\cup \{(v,t)|\exists u\in V_1:uv\in M'\}\\
0&\text{sonst}\end{cases}$$
offensichtlich ein zulässiger Fluss, dessen Wert dem von $M$ entspricht.\\
Andererseits ist aber auch $S:=\delta^+(\{s\}\cup V_1\cup \{v\in V_2|\exists u\in V_1:uv\in M'\}$ ein $(s,t)$-Schnitt, dessen Kapazität dem Wert von $M$ entspricht, denn wenn $c(S)$ größer wäre, so könnten wir ein größeres Matching finden, indem wir alle Kanten aus $S$ in ihrer ungerichteten Form zu $M$ hinzunähmen.\\
Hier wird schon deutlich, wie wir aus größten Flüssen $x$ in $D$, deren Werte alle aus $\{0,1\}$ kommen (im folgenden 0-1-Flüsse genannt), in größte Matchings überführen.\\
Einen beliebigen größten Fluss wandeln wir folgendermaßen in einen größten 0-1-Fluss um:\\
\begin{lstlisting}[mathescape]
for each $su\in A$
  $x_{su}$=0;
for each $vt\in A$
  $x_{vt}$=0;
for each $uv\in A\cap V_1\times V_2$
if 0<$x_{uv}$
  begin
   $x_{uv}$=1;
   $x_{su}$=1;
   $x_{vt}$=1;
   for each $uv'\in A$
     $x_{uv'}$=0;
   for each $u'v\in A$
     $x_{u'v}$=0;  
  end
\end{lstlisting}
Dieser Algorithmus verändert den Wert des ursprünglichen Flusses nicht, da für jeden solchen Bogen $uv$ entweder $su$ oder $vt$ (oder beide) bereits saturiert war, sonst könnte man ja den Fluss vergrößern. Bauen wir uns also einen Schnitt $S$ auf, der $su$ enthält falls dieser Bogen saturiert war oder im anderen Falle $vt$, so erhalten wir einen minimalen Schnitt, der sowohl durch den ursprünglichen Fluss als auch durch den veränderten saturiert wird.
\section*{Aufgabe 22}
\end{document} 
