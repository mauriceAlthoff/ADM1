\documentclass[a4paper,12pt,german]{scrartcl}
%\usepackage{mathtools}
\usepackage[T1]{fontenc}
\usepackage[utf8]{inputenc}
\usepackage[ngerman]{babel}
\usepackage{a4wide}
\usepackage{dsfont,alltt}
\usepackage{amsthm,amssymb,amsmath,ifthen,stmaryrd}
\usepackage{tikz}
\usetikzlibrary{arrows}
\usepackage{listings}
\renewcommand{\labelenumi}{\alph{enumi})}

%opening
\title{Einführung in die lineare und kombinatorische Optimierung\\
Serie 6}
\author{Sven-Maurice Althoff (FU 4745454)\\Michael R. Jung (HU 502133)\\Felix Völker (TU 331834)}

\begin{document}

\maketitle

\section*{Aufgabe 20}
\section*{Aufgabe 21}
  \begin{enumerate}
    \item Diese Aussage ist falsch. Betrachte folgendes Gegenbeispiel:
    \begin{center}
    \begin{tikzpicture}[every node/.style={circle},node distance=40pt]
    \node[draw] (1) {$s$};
    \node[draw] (2)[right of=1] {$u$} edge[<-]node[above]{1} (1);
    \node[draw] (3)[right of=2] {$v$} edge[->,bend left]node[below]{1} (2) edge[<-,bend right]node[above]{1} (2);
    \node[draw] (4)[right of=3] {$t$} edge[<-]node[above]{1} (3);
    \node[draw] (5)[below of=2] {$x$} edge[<-]node[left]{1} (2);
    \node[draw] (6)[below of=3] {$y$} edge[->]node[right]{1} (3)edge[<-]node[below]{1}(5);
    \end{tikzpicture}    
    \end{center}
    Hier können wir alle Bögen saturieren (d.h. $x_a=c_a\,\forall a\in A$) und erhalten einen maximalen Fluss. Insbesondere ist aber $x_{uv}=x_{vu}=1$.

  \item Diese Aussage ist wahr. Seien ein Netzwerk $((V,A),c,s,t)$ und ein maximaler Fluss $x_a,a\in A$ gegeben. Seien $u,v\in V$ mit $x_{uv}\neq0\neq x_{vu}$. O.B.d.A. sei $x_{uv}\geq x_{vu}$ (sonst vertausche $u$ und $v$). Dann ist auch $x'_{a},a\in A$ ein maximaler Fluss mit $x'_a=x_a\,\forall a\in A\setminus\{(u,v),(v,u)\}$ und $x'_{uv}=x_{uv}-x_{vu},x'_{v,u}=0$.\\
  Man sieht, dass für jeden Knoten $v$ gilt: $\sum\limits_{a\in\delta^+(v)}x'_a-\sum\limits_{a\in\delta^-(v)}x'_a=\sum\limits_{a\in\delta^+(v)}x_a-\sum\limits_{a\in\delta^-(v)}x_a$. Dies gilt, da sich für alle Knoten außer $u$ und $v$ nichts ändert und für diese beiden sich beide Summen um den gleichen Betrag ändern. Außerdem bleibt der Fluss auch maximal, da sich die Bilanz auch für $s$ nicht geändert hat. Dieses Verfahren kann man so oft wiederholen, bis die geforderte Bedingung für alle Knotenpaare erfüllt ist, da zwei solche Veränderungen sich gegenseitig nicht beeinflussen.
  \item Diese Aussage ist falsch. Betrachte folgendes Gegenbeispiel:
    \begin{center}
    \begin{tikzpicture}[every node/.style={circle},node distance=60pt]
    \node[draw] (1) {$s$};
    \node[draw] (2)[above right of=1] {$u$} edge[<-]node[above]{9} (1);
    \node[draw] (3)[right of=2] {$v$} edge[<-]node[above]{5} (2);
    \node[draw] (4)[below right of=3] {$t$} edge[<-]node[right]{13} (3);
    \node[draw] (5)[below right of=1] {$x$} edge[<-]node[below]{11} (1) edge[->,bend left]node[above]{8}(3);
    \node[draw] (6)[right of=5] {$y$} edge[->]node[right]{7} (4)edge[<-]node[below]{3}(5)
    edge[<-,bend left]node[right]{4}(2);
    \end{tikzpicture}
    \end{center}
    Hier sind alle Kapazitäten paarweise verschieden, aber sowohl $B:=\delta(\{s\})$ als auch $C:=\delta(\{s,u,x\})$ minimale $(s,t)$-Schnitte mit $c(B)=c(C)=20$. Minimal sind sie, denn es existiert ein Fluss $f_a,a\in A$ mit $f_a=c_a\,\forall a\in A$ und Wert 20. Im Übrigen ist hier sogar jeder $(s,t)$-Schnitt minimal.
    
  \item Diese Aussage ist wahr. Sei $B\subseteq$ ein beliebiger $(s,t)$-Schnitt und $c'_a=\lambda c_a\,\forall a\in A$. Dann ist nach der Veränderung $c'(B)=\sum\limits_{a\in B}c'_a=\sum\limits_{a\in B}\lambda c_a=\lambda\sum\limits_{a\in B}c_a$. Nun ist klar, dass ein minimaler Schnitt auch minimal bleibt, da für beliebige $(s,t)$-Schnitte $S_1,S_2$ mit $c(S_1)\leq c(S_2)$ gilt: $c'(S_1)\leq c'(S_1)\iff \lambda c(S_1)\leq \lambda c(S_2)\stackrel{\lambda>0}{\iff}c(S_1)\leq c(S_2)$.
  \item Diese Aussage ist falsch. Betrachte folgendes Gegenbeispiel:
    \begin{center}
    \begin{tikzpicture}[every node/.style={circle},node distance=60pt]
    \node[draw] (1) {$s$};
    \node[draw] (2)[above right of=1] {$u$} edge[<-]node[above left]{1} (1);
    \node[draw] (3)[below right of=2] {$v$} edge[<-]node[above right]{2} (2);
    \node[draw] (4)[right of=3] {$t$} edge[<-]node[above]{3} (3);
    \node[draw] (5)[below right of=1] {$x$} edge[<-]node[below left]{1} (1) edge[->]node[below right]{2}(3);
    \end{tikzpicture}
    \end{center}
    Hier wäre $B:=\delta(s)$ ein minimaler $(s,t)$-Schnitt mit $c(B)=2$. Erhöhen wir aber jede Kapazität um 2, so erhalten wir folgendes Netzwerk:
        \begin{center}
    \begin{tikzpicture}[every node/.style={circle},node distance=60pt]
    \node[draw] (1) {$s$};
    \node[draw] (2)[above right of=1] {$u$} edge[<-]node[above left]{3} (1);
    \node[draw] (3)[below right of=2] {$v$} edge[<-]node[above right]{4} (2);
    \node[draw] (4)[right of=3] {$t$} edge[<-]node[above]{5} (3);
    \node[draw] (5)[below right of=1] {$x$} edge[<-]node[below left]{3} (1) edge[->]node[below right]{4}(3);
    \end{tikzpicture}
    \end{center}
    Hier ist nun $\delta(s)$ kein minimaler $(s,t)$-Schnitt mehr, denn $c(\delta(s))=6$, \linebreak aber $c(\delta(\{s,u,x,v\}))=5$.\\
    Das Problem bei dieser Veränderung ist, dass  ein $(s,t)$-Schnitt in der Anzahl der beteiligten Kanten skaliert wird.
  \end{enumerate}  
  
\section*{Aufgabe 22}
\section*{Aufgabe 23}

\end{document}
