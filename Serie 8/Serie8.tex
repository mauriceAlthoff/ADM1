\documentclass[a4paper,12pt,german]{scrartcl}
%\usepackage{mathtools}
\usepackage{listings}
\usepackage[T1]{fontenc}
\usepackage[utf8]{inputenc}
\usepackage[ngerman]{babel}
\usepackage{a4wide}
\usepackage{dsfont,alltt}
\usepackage{amsthm,amssymb,amsmath,ifthen,stmaryrd}
\usepackage{tikz}
\usetikzlibrary{arrows}
\usepackage{listings}
\renewcommand{\labelenumi}{\alph{enumi})}

%opening
\title{Einführung in die lineare und kombinatorische Optimierung\\
Serie 8}
\author{Sven-Maurice Althoff (FU 4745454)\\Michael R. Jung (HU 502133)\\Felix Völker (TU 331834)}

\begin{document}

\maketitle

\section*{Aufgabe 29}
Zeige zunächst: $a_{n+2}=a_n-a_{n+1}$:\\
$a_{n+2}=r^{n+2}=r^n\cdot r^2= r^n\left(\frac{\sqrt5-1}2\right)^2 =r^n\left(\frac14(5-2\sqrt5+1)\right)=
r^n\left(\frac14(6-2\sqrt5)\right)$\\
$= r^n\left(\frac12(3-\sqrt5)\right)= r^n\left(1+\frac12(1-\sqrt5)\right)= r^n\left(1-\frac12(\sqrt5-1)\right) =r^n(1-r)$\\
$=r^n-r^{n+1}=a_n-a_{n+1}$.\\
Betrachte den Weg $p=(s32t)$ Setzen wir auf diesem Weg den Fluss gleich 1, so sind die Residualkapazitäten der Bögen $e_1,e_2,e_3$ gleich $(a_1,a_0,0)$. Im Folgenden nennen wir das die \emph{Situation} $(a_1,a_0,0)$.\\
Betrachten wir nun die Pfade (in den entsprechenden Restnetzwerken) 
$$\begin{array}{rcl}
p_1&=&(s1234t)\\
p_2&=&(s321t)\\
p_3&=&p_1\\
p_4&=&(s432t)
\end{array}$$
Sei das Netzwerk in der Situation $(a_n,a_{n+1},0)$, dann können wir den Fluss via $p_1$ um die Residualkapazität  des Bogens $e_2\ (=a_n)$ erhöhen und gelangen in die Situation\linebreak$(a_n-a_{n+1}=a_{n+2},0,a_{n+1})$. Nun können wir allerdings den Fluss via $p_2$ um die Residualkapazität  des Bogens $e_3 (=a_{n+1})$ erhöhen und gelangen in die Situation $(a_{n+2},a_{n+1},0)$. Hier können wir den Fluss via $p_3$ um die Residualkapazität  des Bogens $e_1 (=a_{n+2})$ erhöhen und gelangen in die Situation $(0,a_{n+1}-a_{n+2}=a_{n+3},a_{n+2})$. Zuletzt erhöhen wir den Fluss via $p_4$ nun um die Residualkapazität  des Bogens $e_3 (=a_{n+2})$ und gelangen in die Situation $(a_{n+2},a_{n+3},0)$. Hier sieht man nun, dass wir wieder in einer Situation $(a_{k},a_{k+1},0)$ sind und, da wir zu Beginn via $p$ auch in eine solche können, sieht man nun dass wir für jede natürliche Zahl $n$ eine Folge länger $n$ von augmentierenden Pfaden (z.B $p(p_1,p_2,p_3,p_4)^n$) finden, sodass im aktuellen Restnetzwerk noch augmentierende Pfade vorhanden sind. Folglich gibt es eine unendliche Folge von augmentierenden Pfaden, so dass der Algorithmus von Ford und Fulkerson nicht terminiert. 

\section*{Aufgabe 30}
\section*{Aufgabe 31}
\section*{Aufgabe 32}
Wir modellieren die Aufgabe als bipartites Matching-Problem. Wir haben auf der einen Seite die Geschenke $G_m$ und auf der anderen Seite alle Fertigstellungszeitpunkte $T_n$. Die B\"ogen bekommen eine Kapazit\"at von 1 und einen Kostenkoeffizienten der durch die Kostenfunktion $c(a)=c_g(t_g)$ festgellegt wird. Dann werden die super-Quelle(s) und -Senke(t) hinzugef\"ugt und wir sind fertig.
\begin{center}
   \begin{tikzpicture}[baseline=1.8cm,thick]
    \node[circle,draw] (T1) at (6,2)  {$T_1$};
    \node[circle,draw] (T2) at (6,0)  {$T_2$};
    \node[circle,draw] (T3) at (6,-2) {$T_3$};
    \node[circle,draw] (Tn) at (6,-4) {$T_n$};
    \node[circle,draw] (t) at (8,-1)  {t} edge[<-] (T1) edge[<-] (T2) edge[<-] (T3) edge[<-] (Tn);
    \node[circle,draw] (G1) at (0,2)  {$G_1$} edge[->] (T1) edge[->] (T2) edge[->] (T3) edge[->] (Tn);
    \node[circle,draw] (G2) at (0,0)  {$G_2$} edge[->] (T1) edge[->] (T2) edge[->] (T3) edge[->] (Tn);
    \node[circle,draw] (G3) at (0,-2) {$G_3$} edge[->] (T1) edge[->] (T2) edge[->] (T3) edge[->] (Tn);
    \node[circle,draw] (Gm) at (0,-4) {$G_m$} edge[->] (T1) edge[->] (T2) edge[->] (T3) edge[->] (Tn);
    \node[circle,draw] (s) at (-2,-1) {s} edge[->] (G1) edge[->] (G2) edge[->] (G3) edge[->] (Gm);
   \end{tikzpicture}  
 \end{center}
\end{document} 
