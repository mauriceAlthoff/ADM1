\documentclass[a4paper,12pt,german]{scrartcl}
%\usepackage{mathtools}
\usepackage{listings}
\usepackage[T1]{fontenc}
\usepackage[utf8]{inputenc}
\usepackage[ngerman]{babel}
\usepackage{a4wide}
\usepackage{dsfont,alltt}
\usepackage{amsthm,amssymb,amsmath,ifthen,stmaryrd}
\usepackage{tikz}
\usetikzlibrary{arrows}
\usepackage{listings}
\renewcommand{\labelenumi}{\alph{enumi})}

%opening
\title{Einführung in die lineare und kombinatorische Optimierung\\
Serie 8}
\author{Sven-Maurice Althoff (FU 4745454)\\Michael R. Jung (HU 502133)\\Felix Völker (TU 331834)}

\begin{document}

\maketitle

\section*{Aufgabe 29}
\section*{Aufgabe 30}
\section*{Aufgabe 31}
\section*{Aufgabe 32}
Wir modellieren die Aufgabe als bipartites Matching-Problem. Wir haben auf der einen Seite die Geschenke $G_m$ und auf der anderen Seite alle Fertigstellungszeitpunkte $T_n$. Die B\"ogen bekommen eine Kapazit\"at von 1 und einen Kostenkoeffizienten der durch die Kostenfunktion $c(a)=c_g(t_g)$ festgellegt wird. Dann werden die super-Quelle(s) und -Senke(t) hinzugef\"ugt und wir sind fertig.
\begin{center}
   \begin{tikzpicture}[baseline=1.8cm,thick]
    \node[circle,draw] (T1) at (6,2)  {$T_1$};
    \node[circle,draw] (T2) at (6,0)  {$T_2$};
    \node[circle,draw] (T3) at (6,-2) {$T_3$};
    \node[circle,draw] (Tn) at (6,-4) {$T_n$};
    \node[circle,draw] (t) at (8,-1)  {t} edge[<-] (T1) edge[<-] (T2) edge[<-] (T3) edge[<-] (Tn);
    \node[circle,draw] (G1) at (0,2)  {$G_1$} edge[->] (T1) edge[->] (T2) edge[->] (T3) edge[->] (Tn);
    \node[circle,draw] (G2) at (0,0)  {$G_2$} edge[->] (T1) edge[->] (T2) edge[->] (T3) edge[->] (Tn);
    \node[circle,draw] (G3) at (0,-2) {$G_3$} edge[->] (T1) edge[->] (T2) edge[->] (T3) edge[->] (Tn);
    \node[circle,draw] (Gm) at (0,-4) {$G_m$} edge[->] (T1) edge[->] (T2) edge[->] (T3) edge[->] (Tn);
    \node[circle,draw] (s) at (-2,-1) {s} edge[->] (G1) edge[->] (G2) edge[->] (G3) edge[->] (Gm);
   \end{tikzpicture}  
 \end{center}
\end{document} 
