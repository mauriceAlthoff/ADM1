\documentclass[a4paper,12pt,german]{scrartcl}
%\usepackage{mathtools}
\usepackage{listings}
\usepackage[T1]{fontenc}
\usepackage[utf8]{inputenc}
\usepackage[ngerman]{babel}
\usepackage{a4wide}
\usepackage{dsfont,alltt}
\usepackage{amsthm,amssymb,amsmath,ifthen,stmaryrd}
\usepackage{tikz}
\usetikzlibrary{arrows}
\usepackage{listings}
\renewcommand{\labelenumi}{\alph{enumi})}

%opening
\title{Einführung in die lineare und kombinatorische Optimierung\\
Serie 10}
\author{Sven-Maurice Althoff (FU 4745454)\\Michael R. Jung (HU 502133)\\Felix Völker (TU 331834)}

\begin{document}

\maketitle

\section*{Aufgabe 37}
a) Für alle $i = 1,...,n$ tue:\\
\\
Falls $x_i \ge 0$ oder $a^Tx-b = 0$\\
dann $x ist zulässig$\\
sonst $x ist nicht zulässig$\\
\\
b)  Aus dem Grund, dass alle Gleichungen $A_{i\dot}*x=b_i$ mit $i=1,...,n$ die Seitenflächen des Polyeders definieren, gilt für alle Lösungen, dass sie genau dann zulässig sind, wenn sie auf den Schnittpunkten aller Seitenflächen liegen, da sie sonst die Nebenbedingungen nicht erfüllen würden. Unter der Annahme, dass eine endliche Optimallösung existiert, muss es mindestens eine Seitenfläche $F$ des Polyeders geben, die zu allen anderen Seitenflächen disjunkt ist, da sonst unendlich viele Optimallösungen möglich wären. Die Schnittpunktberechnung würde folgendermaßen eine endliche Menge zulässiger Punkte ermitteln. Da alle Komponenten des Vektor $x$ positiv sein müssen, handelt es sich bei $c$ lediglich um einen konstanten Vektor, dessen Komponenten $x_i$ ins Negative skalieren (wenn $c_i < 0$) oder positiv skalieren (wenn $c_i \ge 0$). Aus diesem Grund muss lediglich der Schnittpunkt gewählt werden, dessen Komponenten am kleinsten ist für positive Skalare $c_i$ und am größten für negative Skalare $c_i$ sind.
\section*{Aufgabe 38}
Sei $x$ die einzige optimale Lösung von $P$, also $x \in P(A,b)$. Da alle Gleichungen $A_{i\dot} * x = b_i$ mit $i = 1,...,n$ Seitenflächen des Polyeders P sind, müssen alle zulässigen Lösungen auf den Seitenflächen des Polyeders liegen, genauer: sie müssen in den Schnittpunkten aller Seitenflächen liegen. Daraus und aus der Annahme, dass (P) nur eine optimale Lösung besitzt, folgt dass es nur einen Schnittpunkt geben kann. Somit sind alle zulässigen Lösungen auch optimale Lösungen.
\section*{Aufgabe 39}
\section*{Aufgabe 40}
"Frohes Neues Jahr!"
\end{document} 
