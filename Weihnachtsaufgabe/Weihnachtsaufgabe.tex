\documentclass[a4paper,12pt,german]{scrartcl}
%\usepackage{mathtools}
\usepackage{listings}
\usepackage[T1]{fontenc}
\usepackage[utf8]{inputenc}
\usepackage[ngerman]{babel}
\usepackage{a4wide}
\usepackage{dsfont,alltt}
\usepackage{amsthm,amssymb,amsmath,ifthen,stmaryrd}
\usepackage{tikz}
\usetikzlibrary{arrows}
\usepackage{listings}
\renewcommand{\labelenumi}{\alph{enumi})}

%opening
\title{Einführung in die lineare und kombinatorische Optimierung\\
Weihnachtsaufgabe}
\author{Sven-Maurice Althoff (FU 4745454)\\Michael R. Jung (HU 502133)\\Felix Völker (TU 331834)}

\begin{document}

\maketitle

\section*{a)}
Der Graph $D_4$ ist nicht zusammenhängend, er besitzt 205 Zusammenhangskomponenten. 
\section*{b)}
Der größte Grad eines Knotens des Graphen $D_4$ ist 16. \textit{Maus} ist das einzige Wort, dessen Knoten diesen Grad besitzt und besitzt daher die größte Nachbarschaft. Es existieren viele isolierte Knoten, meist die Knoten dessen Worte Symbole aus anderen Sprachen besitzen(bspw. Apostroph) besitzen oder eventuell Namen repräsentieren. Beispiele: \textit{Ulks}, \textit{Ryti}, \textit{Rácz}, \textit{Onyx}, \textit{Ezio}, \textit{Abbé}
\section*{c)}
Es existiert nur eine Zusammenhangskomponente mit mindestens 20 Knoten und zwar diejenige, die den Knoten \textit{Aale} enthält. 
\section*{d)}
\end{document} 
